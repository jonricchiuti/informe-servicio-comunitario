\chapter*{Descripción del problema}
\addcontentsline{toc}{chapter}{Descripción del problema}

\section*{Descripción de la comunidad}
\addcontentsline{toc}{section}{Descripción de la comunidad}

Se benefician todos los habitantes de la ciudad de Caracas. Esto se debe, a que el mantenimiento forestal que se lleva a cabo en la reserva ecológica de la USB, mantiene limpias las aguas que fluyen hacia las cuencas de la Mariposa y la Guairita, estas aguas surten a la ciudad de Caracas. También, el mantenimiento del área constituye un espacio de purificación para el aire de la capital. Por otro lado se realizan actividades de concientización ambiental como los son las ``ecorutas'' y las jornadas de plantación en las que participan los caraqueños.

\section*{Antecedentes del proyecto}
\addcontentsline{toc}{section}{Antecedentes del proyecto}

La Universidad Simón Bolívar se encuentra en una zona geográfica donde tanto las fuertes y constantes precipitaciones en período de lluvia como los incendios forestales generan deforestación. Para combatirla, en los años 70 se plantaron aproximadamente 45 hectáreas de pinos en la zona. En esa época se plantaron dos especies de pino, el pino Caribe y el de Pátula. Estos pinos se caracterizan por su rápido crecimiento, lo cual era necesario en aquella época para frenar la deforestación. Sin embargo al momento de realizar la plantación, los pinos fueron dispuestos muy cerca unos de otros. Esto hizo que sea difícil que la luz penetre al interior del bosque y de esta forma limitando el crecimiento de plantas bajas. Actualmente se busca la reforestación del bosque con especies nativas para de esta forma poblar el bosque con plantas de la región.