\chapter*{Introducción}
\addcontentsline{toc}{chapter}{Introducción}

El medio ambiente es el conjunto de componentes físicos, biológicos, sociales y culturalesque son capaces de producir un impacto directo o indirecto en los seres vivos. El ser humano ha sabido aprovechar el ambiente para evolucionar y desarrollarse. Sin embargo, esta necesidad de crecer está impactado negativamente sobre el ambiente, que nos proporciona vida. Con la tala de árboles, los desperdicios industriales entre otras cosas, el ser humano está acabando con hábitats enteros.
\\  

Uno de los principales problemas derivados de la explotación del medio ambiente, es la disminución de las áreas verdes. En nuestro planeta hay cada día menos plantas. Son organismos vivientes que producen algo más que oxígeno y proporcionan más que un hábitat para otros seres vivos. Las plantas son seres vivos capaces de transformar la energía solar y convertirla en alimento. Todo el reino animal depende de ellas y esto incluye al ser humano. Por lo tanto la explotación incesable del medio ambiente afecta de forma indirecta pero muy importante al ser humano. Sin él no somos capaces de vivir.
\\

El servicio comunitario de gurdabosques de la Universidad Simón Bolívar representa una oportunidad de contribuir con el medio ambiente a través de las plantas. En el servicio, se realizan actividades de cultivo y plantación, con el fin de reforestar. La reforestación entre otras muchas cosas aumenta la fertilidad de los suelos. Ayuda a reducir el flujo rápido de las aguas de lluvia, esto mejora la calidad de las aguas superficiales (ríos, lagunas, entre otras) ya que se reduce la entrada de sedimentos a las mismas. La cobertura vegetal constituye un medio para la purificación del aire (absorción de carbono) y restauración de la atmósfera.
\\

Cabe destacar que la labor de los miembros del servicio comunitario va más allá de la reforestación. También se encargan de sembrar conciencia en la comunidad por medio de diferentes actividades. De esta forma la comunidad puede convertirse en agente activo de cambio que genere un impacto positivo sobre el medio ambiente.