\chapter*{Justificación del servicio comunitario}
\addcontentsline{toc}{chapter}{Justificación del servicio comunitario}

El programa de guardabosques de la USB busca lograr una sensibilización ambiental en los miembros de la comunidad. Demostrar a la comunidad de que es capaz de contribuir en la mejora de sus condiciones ambientales. Por lo anterior, las actividades que se realizan en el programa de guardabosques muchas veces incluyen de la participación activa de la comunidad.
\\

Una de las actividades principales de guardabosques es el mantenimiento forestal de hectáreas de bosques que circundan la Universidad Simón Bolívar. Se realizan jornadas de reforestación en las que se trasplantan cientos y hasta miles de árboles en distintos sectores del valle de Sartenejas. La reforestación tiene un efecto positivo sobre las comunidades aledañas ya que el valle de Sartenejas purifica el aire circundante. También tiene un impacto positivo sobre todas las comunidades que reciben agua del embalse de La Mariposa, ya que limpia una de las principales entradas de agua al mismo.