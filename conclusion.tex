\chapter*{Conclusiones y recomendaciones}
\addcontentsline{toc}{chapter}{Conclusiones y recomendaciones}

De las experiencias adquiridas mediante a la relización del servicio comunitario en guardabosques de la Universidad Simón Bolívar, se puede decir que es de gran importancia que se realicen las estas labores. No solo por el impacto directo que estas tienen sobre el medio ambiente, también porque a través de la concientización y sensibilización ambiental, los miembros de la comunidad son capaces de entender que los árboles no nos necesitan, los seres humanos los necesitamos a ellos. Al entender esto, se abren las puertas para un desarrollo sustentable.
\\

En guardabosques hay mucho trabajo que hacer y toda intención de trabajar debe ser bien recibida. Se da mucho el caso que el servicio no abre algún día o en alguna hora porque no hay un quórum mínimo. Sin embargo hay muchas actividades diarias que se pueden realizar y que no necesitan de una cantidad mínima de personas para hacerla (con una basta). Se debe buscar una forma de cuantificar el trabajo de tal manera que no se necesite de una supervisión estricta sobre algunas actividades. Por ejemplo, si es necesario realizar una actividad de trasplante y una sola persona tiene la disposición de trabajar en un día u hora en particular. Se le podría permitir trabajar y al momento de que se realice el reporte de las horas, estas se puedan constatar por medio de algún registro fotográfico de la actividad. Esto es, si se reportan cuatro horas de trabajo más vale que se hayan trasplantado unas 20 plantas aproximadamente. Para implementar algo como lo descrito anteriormente es necesario llevar registro de cuanto se tardan las personas realizando trasplantes.